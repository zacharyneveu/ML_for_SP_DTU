% Geometry, font
\documentclass[12pt, letter]{article}
\usepackage[margin=0.8in]{geometry}
\usepackage[T1]{fontenc}
\usepackage{fourier}
\usepackage{titling}
\setlength{\droptitle}{-5em} 
\usepackage[parfill]{parskip}
\usepackage{graphicx}
\graphicspath{{imgs/}}
\usepackage{hyperref}

% Math stuff
\usepackage{amssymb}
\usepackage{amsmath}
\usepackage{bm}

% Code Highlighting
\usepackage{minted}
\usemintedstyle{solarizedlight}

\author{Zach Neveu}
\title{ Intro Lecture Notes }

\begin{document}
\maketitle

\section{Course Intro}%
\label{sec:course_intro}
\begin{itemize}
	\item Using DTU Learn instead of DTU Inside
	\item Goal: learn fundamental sp methods, and code in Matlab/Python
	\item Should be able to create a modern signal processing system with ML by the end
	\item 4 hours of hw/week
	\item each class 1 hour of lecture, 3 hours of exercises
	\item Exercises as live scripts/Jupyter notebooks
\end{itemize}

\section{Course Plan}%
\label{sec:course_plan}

\begin{itemize}
	\item 3 parts
	\item Conventional DSP
	\item Linear Methods
	\item Non-linear methods - hmms neural nets
\end{itemize}

\section{Technical Lecture}%
\label{sec:technical_lecture}
\begin{itemize}
	\item Traditional SP - doesn't care much about input content
	\item Traditional ML - not particularly friendly for time series of signals
	\item MLSP - combines the two
	\item Example Areas
	\begin{itemize}
		\item sparsity-aware learning - compression
		\item Information-theoretic learning
		\item Adaptive filtering
		\item Sound processing
		\item Images/Videos
		\item Telecommunications
		\item Sensors
	\end{itemize}
\end{itemize}

\section{Connection}%
\label{sec:connection}
\begin{itemize}
	\item Simplest connection - $h(t)$ can be a classifier etc. ML is just a specific kind of of non-linear processing.
	\item Another idea: signal processing is how to get a small amount of features from a large amount of data
\end{itemize}

\end{document}
